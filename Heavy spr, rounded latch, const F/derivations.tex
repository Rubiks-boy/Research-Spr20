\documentclass[12pt]{article}
\usepackage{geometry}
\usepackage{enumerate}
\usepackage{amsthm}
\usepackage{amsmath}
\usepackage{amssymb}
\usepackage{etoolbox}
\usepackage{graphicx}
\usepackage{framed}
\usepackage{tikzsymbols}
\usepackage{hyperref}

\begin{document}
\begin{center}
    Ideal F-v motor, Heavy Spring, Rounded Latch @ Constant Force \\
    30 Jan 2020
\end{center}

\noindent \textbf{Parameters:}
\newline
Motor: $F_{\text{max}}$, $d$, $v_{\text{max}}$ \newline
Spring: $m_{\text{spr}}$ \newline
Latch: $F_l$, $m_l$, $v_{\text{l0}}$, $R$ \newline
Projectile: $m$ \newline
Timing: $t_l$ (time of latch releasing), $t_{\text{to}}$ (time of projectile taking off)
\newline
Non-dimensionalized variables: see pg. 7, supplemental.
\newline \newline
\noindent \textbf{Prior to latch release} ($t \leq t_l$)
\newline
Latch moves in y direction, projectile in x direction (like supplemental).
\newline
Position of the rounded latch, w/r/t time (assume $y(0) = 0$) for a constant force:
\[ 
    y(t) = v_0 t + \frac{1}{2} \frac{F_{\text{max}}}{m_l}\ t^2
\]
Position of projectile while in contact with latch:
\[
    x(t) = R \left(1 - \sqrt{1-\left( \frac{y(t)}{R} \right) ^2}\right)
\]
Velocity of projectile, in terms of $y(t)$:
\[
    v(t) = \frac{y(t)y'(t)}{R\sqrt{1 - \left(\frac{y(t)}{R}\right)^2}}
\]
Acceleration:
\[
    a(t) = \frac{1}{R \sqrt{1 - \left( \frac{y(t)}{R} \right) ^2}} \left( y'(t)^2 + y(t)y''(t) + \frac{y(t)^2 y'(t)^2}{R^2 - y(t)^2}\right)
\]
\newline
Game-plan: By leaving these equations in terms of $y(t)$, we can declare $y(t)$ as a separate function in Matlab. This hopefully will also allow changing $y(t)$ easily, if needed (e.g. for testing/debugging).
\newline \newline
Sanity checking: Plugging in $y(t) = v_0$ in each of these equations yielded identical results to equations 27-29 in supplemental text.
\newline \newline
Non-dimensionalizing: Yields exact same equations (with tildas)
\newline 
\newline
\noindent \textbf{Solving for time of latch release $t_l$}
\newline
Using the energy argument of section 6 of supplemental,
\[
    \frac{1}{2}(1-\tilde{x})^2 + \frac{1}{2}\left(\tilde{m} + \frac{\tilde{m}_{\text{spr}}}{3} \right)\tilde{v}^2 - \int_0^{\tilde{x}} \! f_{\text{latch}}(x') \, \mathrm{d}x' = constant. 
\]
\[
    \rightarrow \left(\tilde{m} + \frac{\tilde{m}_{\text{spr}}}{3} \right)\tilde{a}(\tilde{t_l}) = 1 - \tilde{x}(\tilde{t_l})
\]
We can substitude the equations for $x(t)$ and $a(t)$ above into this expression, and solve for $t_l$.
\newline 
\newline
\noindent \textbf{Solving for velocity at takeoff $v_{\text{to}}$:}
\newline
Based on conservation of energy: $\tilde{U}(\tilde{t_l}) + \tilde{KE}(\tilde{t_l}) = \tilde{KE}(\tilde{v_{\text{to}}})$), we get equation 39 in supplemental:
\[
    \tilde{v_{\text{to}}} = \sqrt{\tilde{v_l}^2 + \frac{1}{\tilde{m} + \frac{\tilde{m}_{\text{spr}}}{3}} (1 - \tilde{x_l})^2}
\]
where $\tilde{v}_l$ and $\tilde{x}_l$ are the equations for $\tilde{v}(\tilde{t})$ and $\tilde{x}(\tilde{t})$ above, evaluated at $\tilde{t}_l$.
\newline 
\newline
\noindent \textbf{After latch release $t_l < t < t_{\text{to}}$:}
\newline
Equations 61-67 in supplemental.
\newline
\newline
\noindent \textbf{After takeoff $t > t_{\text{to}}$:}
\[
    x(t) = v_{\text{to}}t
\]
\[
    v(t) = v_{\text{to}}
\]

\newpage
\begin{center}
    Ideal F-v motor, Heavy Spring, Rounded Latch @ Constant Force \\
    27 Feb 2020
\end{center}

\noindent
Before, all derivations assumed an ideal $k$ value for the spring; i.e. $k_{ideal} = F_{max} / d$. Now, we will mdoify these derivations for a non-ideal spring stiffness. \newline

\noindent \textbf{New Parameters:}
\newline
$E$, young's modulus (stiffness of material) \newline
$A$, cross-sectional area of spring \newline
$L$, rest length of the spring \newline
$k = \frac{EA}{L}$ \newline

\noindent \textbf{Dimensionless values:} 
\newline
$\tilde{k} = \frac{k}{k_{ideal}} = \frac{kd}{F_{max}}$
\newline
\newline
\noindent \textbf{Solving for time of latch release $t_l$}
\newline
Modifying the potential energy to include $\tilde{k}$,
\[
    \frac{1}{2}\tilde{k}(1-\tilde{x})^2 + \frac{1}{2}\left(\tilde{m} + \frac{\tilde{m}_{\text{spr}}}{3} \right)\tilde{v}^2 - \int_0^{\tilde{x}} \! f_{\text{latch}}(x') \, \mathrm{d}x' = constant. 
\]
\[
    \rightarrow \left(\tilde{m} + \frac{\tilde{m}_{\text{spr}}}{3} \right)\tilde{a}(\tilde{t_l}) = \tilde{k} \biggl( \tilde{x}_{max} - \tilde{x}(\tilde{t_l}) \biggr)
\]

where 
\[
    \tilde{x}_{max} = \frac{x_{max}}{d} = \frac{F_{max} / k}{d} = \frac{1}{\tilde{k}}.
\]
\newline 
\newline
\noindent \textbf{Solving for velocity at takeoff $v_{\text{to}}$:}
\newline
\[
    \tilde{v_{\text{to}}} = \sqrt{\tilde{v_l}^2 + \frac{\tilde{k}}{\tilde{m} + \frac{\tilde{m}_{\text{spr}}}{3}} (\tilde{x}_{max} - \tilde{x_l})^2}
\]
\newline 
\newline
\noindent \textbf{After latch release $t_l < t < t_{\text{to}}$:}
\newline
In equation 61, switch in maximum position $\tilde{x}_{max}$ and add spring constant $\tilde{k}$ to give
\[
    \tilde{x}(\tilde{t}) = \tilde{x}_{max} - \tilde{v}_{to} \sqrt{\frac{\tilde{m} + \frac{\tilde{m}_{spr}}{3}}{\tilde{k}}} \cos{\biggl(\sqrt{\frac{\tilde{k}}{\tilde{m} + \frac{\tilde{m}_{spr}}{3}}} \ \tilde{t} + \tilde{\phi} \biggr)} \]
In equations 62-67 in supplemental, replace $\tilde{m} + \frac{\tilde{m}_{spr}}{3}$ terms with $(\tilde{m} + \frac{\tilde{m}_{spr}}{3}) / \tilde{k}$.
\newline

\end{document}