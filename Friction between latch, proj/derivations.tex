\documentclass[12pt]{article}
\usepackage{geometry}
\usepackage{enumerate}
\usepackage{amsthm}
\usepackage{amsmath}
\usepackage{amssymb}
\usepackage{etoolbox}
\usepackage{graphicx}
\usepackage{framed}
\usepackage{tikzsymbols}
\usepackage{hyperref}

\begin{document}
\begin{center}
    Friction between latch and projectile \\
    9 Apr 2020
\end{center}


\noindent This file only includes the derivations for while the projectile is in contact with the latch.
\newline
\noindent \textbf{Using $\Sigma F = ma$}
\newline
Latch moves in x direction, projectile in y direction.
\newline
For the latch,
\[ \tag{1}
    \hat{x}: F_l + N \sin{\theta} - \mu_k N \cos{\theta} = m_l a_l
\]
Solving for N
\[\tag{2}
    N = \frac{m_l a_l - F_l}{\sin{\theta} - \mu_k \cos{theta}}
\]
\newline
For the projectile,
\[\tag{3}
    \hat{y}: F_{spr} + \mu_k N \cos{\theta} - N \cos{\theta} = m a
\]
Solving for N
\[\tag{4}
    N = \frac{F_{spr} - ma}{\cos{\theta} + \mu_k \sin{\theta}}
\]
\newline
\noindent \textbf{Putting accelerations $a$, $a_l$ in terms of $\theta$}
\newline
\begin{align}
    x & = r \sin{\theta} \tag{5} \\
    a_l & = r \cos{\theta} \  \ddot{\theta} - r \sin{\theta} \  \dot{\theta} \tag{6} \\
    y & = r (1 - \cos{\theta}) \tag{7} \\
    a & = r \sin{\theta} \  \ddot{\theta} - r \cos{\theta} \  \dot{\theta} \tag{8}
\end{align}
\newline
\noindent \textbf{Eliminating normal force N}
\newline
Plugging equations (6), (8) into (2), (4) (respectively), we find
\begin{align}
    N = \frac{m_l (r \cos{\theta} \  \ddot{\theta} - r \sin{\theta} \  \dot{\theta}) - F_l}{\sin{\theta} - \mu_k \cos{\theta}} \tag{9} \\
    N = \frac{F_{spr} - m (r \sin{\theta} \  \ddot{\theta} - r \cos{\theta} \  \dot{\theta})}{\cos{\theta} + \mu_k \sin{\theta}} \tag{10} \\
\end{align}
\newline
Setting (9) and (10) equal to each other,
\begin{align}
    (\cos{\theta} & + \mu_k \sin{\theta}) \bigl( m_l (r \cos{\theta} \  \ddot{\theta} - r \sin{\theta} \  \dot{\theta}) - F_l \bigr) \notag \\
    & = (\sin{\theta} - \mu_k \cos{\theta}) \bigl(F_{spr} - m (r \sin{\theta} \  \ddot{\theta} - r \cos{\theta} \  \dot{\theta}) \bigr) \tag{11}
\end{align}
\newline
Dividing by $\cos{\theta}$, we can place this into the form $A + B \dot{\theta} + C \ddot{\theta}$
\begin{align}
    (\tan{\theta} & - \mu_k) F_{spr} + F_l(1+\mu_k \tan{\theta}) \notag \\
    & + [\mu_k mr \cos{\theta} - mr \sin{\theta} + m_l r \sin{\theta} + \mu_k m_l r \sin{\theta} \tan{\theta}]\ \dot{\theta} \notag \\
    & + [\mu_k mr \sin{\theta} - mr \sin{\theta} \tan{\theta} - m_l r \cos{\theta} - \mu_k m_l r \sin{\theta}]\ \ddot{\theta} = 0 \tag{12}
\end{align}
\newline
\noindent \textbf{Substitutions to put eqn (12) in terms of $y$}
\newline
Using a rounded latch,
\[\tag{13}
    y = r - \sqrt{r^2 - x_l^2}
\]
\[\tag{14}
    y' = \frac{dy}{dx_l} = \frac{x_l}{\sqrt{r^2 - x_l^2}}
\]
Since $x_l = r \sin{\theta}$,
\[
    y' = \frac{r \sin{\theta}}{\sqrt{r^2 (1 - \sin^2{\theta})}} = \frac{r \sin{\theta}}{r \cos{\theta}}
\]
\[\tag{15}
    y' = \tan{\theta}
\]
which matches our expectation that $\frac{dy}{dx_l} = \tan{\theta}$.
\newline
\begin{align}
    y'' & = \frac{d}{dx_l} \biggl( x_l (r^2 - x_l^2)^{-1/2} \biggr) \notag \\
    & = (r^2 - x_l^2)^{-1/2} + x_l \biggl( \frac{-1}{2} (r^2 - x_l^2)^{-3/2} (-2 x_l) \biggr) \notag  \\
    & = \frac{1}{r \sqrt{1 - \sin^2{\theta}}} + \frac{x_l^2}{(r^2 - x_l^2)^{3/2}} \notag \\
    & = \frac{1}{r \cos{\theta}} + \frac{1}{r} \ \frac{r^2 \sin^2{\theta}}{r^2 (1- \sin^2{\theta})^{3/2}} \notag \\
    & = \frac{1}{r \cos{\theta}} + \frac{1}{r} \ \frac{\sin^2{\theta}}{\cos^3{\theta}} \notag \\
    & = \frac{1}{r \cos{\theta}} (1 + \tan^2{\theta}) \notag \\
    y'' & = \frac{1}{r} \sec^3{\theta} \tag{16}
\end{align}
\newline
Since $y = r - r \cos{\theta}$, $\frac{dy}{d\theta} = r \sin{\theta}$ and 
\begin{align}
    \dot{y} = \frac{dy}{dt} = \frac{dy}{d\theta} \frac{d\theta}{dt} = r \sin{\theta} \ \dot{\theta} \tag{17}
\end{align}
\newline
Similarly,
\begin{align}
    \ddot{y} & = \frac{d}{dt}\biggl( r \sin{\theta} \frac{d\theta}{dt}\biggr) \notag \\
    \ddot{y} & = r \cos{\theta} \ \dot{\theta}^2 + r \sin{\theta} \ \ddot{\theta} \tag{18}
\end{align}
\newline
\noindent \textbf{Compare to diff. eqn. w/r/t $y$}
\newline
Prof. Ilton's derivation:
\begin{align}
    \biggl[ F_{spr} & (\mu_k - y_l') - F_l(\mu_k y_l' + 1) \biggr] - \biggl[ m_l (1 + \mu_k y_l') \frac{y_l''}{y_l'^3} \biggr] \dot{y} \notag \\
    & + \biggl[ m (y_l' - \mu_k) + \frac{m_l}{y_l''} (1 + \mu_k y_l') \biggr] \ddot{y} \notag \\
    & = A + B \dot{\theta} + C \ddot{\theta} = 0 \tag{19}
\end{align}
\newline
Starting with the A component of eqn (19) and making the substitutions from above,
\begin{align*}
    F_{spr} & (\mu_k - y_l') - F_l(\mu_k y_l' + 1) \\
    & = F_{spr} (\mu_k - \tan{\theta}) - F_l(\mu_k \tan{\theta} + 1)
\end{align*}
which matches the A component of eqn (12), off by a factor of $-1$.
\end{document}